\documentclass[14pt,xcolor=pdftex,dvipsnames,table]{beamer}

% Specify theme
\usetheme{Madrid}
% See deic.uab.es/~iblanes/beamer_gallery/index_by_theme.html for other themes
\usepackage{caption}

% Specify base color
\usecolortheme[named=OliveGreen]{structure}
% See http://goo.gl/p0Phn for other colors

% Specify other colors and options as required
\setbeamercolor{alerted text}{fg=Maroon}
\setbeamertemplate{items}[square]

% Title and author information
\title{Capital Flow and the Real Exchange Rate}
\author{Rob Hayward}


\begin{document}

\begin{frame}
\titlepage
\end{frame}

\begin{frame}{Outline}
\tableofcontents
\end{frame}

\section{Introduction}
\begin{frame}{Introduction}
\begin{itemize}[<+-| alert@+>]
\item There are prolonged and significant deviations from \textbf{PPP}
\item How far are deviations from PPP a function of \textbf{International Capital Flows?}
\item Are some \textbf{capital flows} more powerful than others? 
\item Are some \textbf{exchange rates} more vulnerable than others?  
\end{itemize}
\end{frame}

\begin{frame}{Microstructure}
\framesubtitle{Order Flow at the macro level}
\begin{itemize}[<+-| alert@+>]
\item Evans and Lyons (2002) add order flow to conventional exchange rate model
\item \begin{equation}
 \Delta s_t=f(i,m,z,)+g(X,I,Z)+\epsilon_t
\end{equation}
\item Construct model of  \textbf{international capital flows} 
\item \textbf{Signed} order flow is deliberate and is offset by a \textbf{passive} balancing flow
\end{itemize}
\end{frame}


\section{Measuring Capital Flows}
\begin{frame}{Measuring Capital Flows}
\begin{itemize}[<+-| alert@+>]
\item Capital flow is hard to measure
\item Portfolio balance models (Kouri and Porter, 1974) use bonds as a proxy
\item However, \textbf{international agencies (BIS, IMF)} as well as \textbf {national statistical agencies} and \textbf{private economists} have sought to improve the measurement of capital flows
\item Monthly capital flows and measure of returns with equity flows significant while bond flows are not (Siouronis 2008)
\end{itemize}
\end{frame}


\begin{frame}{Capital Flow Series}
\begin{itemize}[<+-| alert@+>]
\item RTWI - Real Trade Weighted Index
\item CNB - Cumulative net bonds per GDP
\item CNE - Cumulative net equity per GDP
\item CNFDI - Cumulative net equity per GDP
\item COT - Cumulative Official Treasuries per GDP
\item Spread - Three month interest rate spread
\item S1 - CFTC FX Derivative Positions (non-commercial per open interest)
\end{itemize}
\end{frame}

\begin{frame}
\frametitle{Capital Flow Series}
\framesubtitle{Descriptive Statistics}
\begin{center}
\rowcolors{1}{OliveGreen!20}{OliveGreen!5}
 \begin{tabular}{l r r r r r}
Series & Units & Mean & Median &  Max & Min \\
\hline
RTWI & index & 91.24 & 89.26 & 115.96 & 78.44 \\
CNB & \% GDP & 6.00 & 5.07 & 15.97 & -0.84\\
CNE & \% GDP & 0.23 & -0.00 & 3.28 & -1.81  \\
CNFDI & \% GDP & -0.67 & -1.29 & 8.91 & -11.48\\
COT & \% GDP & 0.88 & 8.32 & 4.47 & -1.27 \\
SPREAD &  pp & -0.20 & -0.07 & 3.38 & -4.99  \\
S1 &  NC/OI & 47.0 & 3.00 & 70.0 & -69.0\\
\hline
\end{tabular}
\end{center}
\end{frame}

\begin{frame}
\includegraphics<1>[width=12cm, height=8cm]{ts}
\end{frame}

\begin{frame}
\includegraphics<1>[width=12cm, height=8cm]{bondflow}
\end{frame}

\begin{frame}
\includegraphics<1>[width=12cm, height=8cm]{fdiflow}
\end{frame}

\begin{frame}{Vector Auto Regression}
\structure{Structural Equation}
$Bx_{t}=\Gamma_{0}+\Gamma_{1}x_{t-1}+\epsilon_{t}$
\pause

\begin{block}{}
Where: $x_t$ is a vector of endogenous variables; matrix $B$ contains the coefficients for the contemporaneous relationships between the endogenous variables; $\Gamma_0$ contains exogenous variables such as a constant, trend or seasonal; $\epsilon$ is a vector of errors that are assumed to IID; and $\Gamma_1$ are the parameters to be estimated. 
\end{block}
\pause
\end{frame}

\begin{frame}{Vector Auto Regression 2}
\structure{Standard Form of VAR}
$x_t=A_0+A_1x_{t-1}+e_t$
\pause

\begin{block}{}
Multiplying $Bx_{t}=\Gamma_{0}+\Gamma_{1}x_{t-1}+\epsilon_{t}$
through by $B^{-1}$ will give $x_t=A_0+A_1x_{t-1}+e_t$
With $A_0=B^{-1}\Gamma_0$, $A_1=B^{-1}\Gamma_1$ and $e_t=B^{-1}\epsilon_t$. 
\end{block}
\pause
\vskip1cm
Allows OLS to be used
\end{frame}

\begin{frame}{Identification}
\structure{Restrictions}
$K\frac{(K-1)}{2}$ restrictions are needed
\pause

\begin{block}{}
Restrictions on the B matrix can be imposed in an arbitrary fashion (Make B a \emph{lower triangle}) or by using economic theory and other prior information.  
\end{block}
\pause
\vskip1cm
Each method is used here.
\vskip1cm
Results are compared
\end{frame}

\begin{frame}
\frametitle{SVAR Restrictions}
\framesubtitle{NA is estimated}
\begin{center}
\rowcolors{1}{OliveGreen!20}{OliveGreen!5}
\begin{tabular}{lrrrrrrrrr}  
  \hline
 &  CNB & CNE & CNFDI & COT & RTWI & SP & S1 \\ 
  \hline
  CNB & 1 & NA & 0 & 0 & 0 & NA & 0\\ 
  CNE & NA & 1 & NA & 0 & NA & 0 & NA\\ 
  CNFDI & 0 & NA & 1 & 0 & NA & 0 & 0\\ 
  COT & NA & 0 & 0 & 1 & NA & 0 & NA \\ 
   RTWI & 0 & NA & NA & NA & 1 & NA & NA\\ 
  SPR'D & NA & 0 & 0 & 0 & NA & 1 & 0\\ 
  S1 & 0 & NA & 0 & NA & NA & 0 & 1\\ 
   \hline
\end{tabular}
\end{center}
\end{frame}

\begin{frame}{Impulse Response Functions}
What is the effect of an innovation or shock? 
\begin{equation}
x_t=\mu+\sum_{i=0}^{i=n}\frac{A_t^i}{1-b_{12}b_{21}}
\begin{bmatrix} 1 & -b_{12} \\ -b_{21} & 1
\end{bmatrix}
\end{equation}
For n periods
\end{frame}


\begin{frame}
\includegraphics<1>[width=12cm, height=8cm]{irfspread2}
\end{frame}

\begin{frame}
\includegraphics<1>[width=12cm, height=8cm]{irfcot2}
\end{frame}

\begin{frame}
\includegraphics<1>[width=12cm, height=8cm]{irfcnb2}
\end{frame}

\begin{frame}
\includegraphics<1>[width=12cm, height=8cm]{irfs2}
\end{frame}

\begin{frame}
\includegraphics<1>[width=12cm, height=8cm]{irfcne2}
\end{frame}

\section{Findings}
\begin{frame}{Findings}
\begin{itemize}[<+-| alert@+>]
\item \textbf{Speculative flow} seems to have a significant and persistent effect on the real exchange rate
\item There is a positive relationship between \textbf{interest rate differentials} and the US dollar
\item \textbf{FDI}, \textbf{Bond flow} and \textbf{Equity flows} seem to have minimal influence on the real exchange rate
\item \href{https://github.com/RobHayward/SVARdoc}{https://github.com/RobHayward/SVARdoc}
\end{itemize}
\end{frame}

\begin{frame}{References}
\textbf{Lyons and Evans} (2002), 'Order Flow and Exchange Rate Dynamics', \emph{Journal of Political Economy}, 110 (1)\\
\textbf{Kouri and Porter} (1974), 'International Capital Flows and Portfolio Equilibrium', \emph{Journal of Political Economy}, 82\\
\textbf{Sims} (1980), 'Macroeconomics and Reality', \emph{Econometrica} 48(1)
\end{frame}
\end{document}
